% Paket um vordefinierte Texte (z.B. "Inhaltsverzeichnis") auf Deutsch zu übersetzen
\usepackage[ngerman]{babel}

% Paket um Schriftarten festzulegen (für XeLaTeX)
\usepackage{fontspec}

% Schriftart festlegen
\usepackage{lmodern}
\renewcommand{\familydefault}{\sfdefault}

\setmonofont[]{Hack}

% Paket um Grafiken (JPG, PNG, PDF) einzubinden
\usepackage{graphicx}
\graphicspath{{pictures/}}

% Paket für Tabellenlayout
\usepackage{booktabs}

% Paket für Syntaxhighlighting für inline code
\usepackage[minted]{tcolorbox}
\usemintedstyle{manni}
\tcbuselibrary{listingsutf8}
\newtcolorbox{codeblock}[1][]{%
  colback=gray!5,       % Hintergrundfarbe
  colframe=gray!60,     % Rahmenfarbe
  boxrule=0.4pt,        % Rahmenstärke
  arc=2mm,              % abgerundete Ecken
  outer arc=2mm,
  boxsep=2pt,           % Innenabstand
  left=2pt, right=2pt, top=2pt, bottom=2pt,
  listing only,
  minted options={fontsize=\small, breaklines, #1}
}

% Paket für Zeilenabstand
\usepackage{setspace}

% Paket für Hyperlinks
\usepackage[
  colorlinks=true,
  linkcolor=black,      % For internal links (like table of contents)
  citecolor=black,      % For citations
  urlcolor=blue         % For URLs
]{hyperref}

% Paket für korrekte Anführungszeichen
\usepackage{csquotes}

% Paket für selbst definierte Kopf- und Fusszeilen
\usepackage{scrlayer-scrpage}

% Pakte für Zitate und Bibliografie
\usepackage[style=ieee, citestyle=numeric-comp]{biblatex}
\addbibresource{literature.bib}

% Paket zum Erzeugen von Platzhaltertext
\usepackage{blindtext}