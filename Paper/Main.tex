\documentclass[parskip=full]{scrreprt}
% Paket um vordefinierte Texte (z.B. "Inhaltsverzeichnis") auf Deutsch zu übersetzen
\usepackage[ngerman]{babel}

% Paket um Schriftarten festzulegen (für XeLaTeX)
\usepackage{fontspec}

% Schriftart festlegen
\usepackage{lmodern}
\renewcommand{\familydefault}{\sfdefault}

\setmonofont[]{Hack}

% Farben
\usepackage{xcolor}
\definecolor{TeLicolor}{RGB}{90, 95, 110}
\definecolor{plusGreen}{RGB}{0, 150, 0}
\definecolor{minusRed}{RGB}{200, 0, 0}
\definecolor{boxBG}{RGB}{230, 230, 230}
\definecolor{boxFrame}{RGB}{50, 50, 50}

% Für Untertitel
\usepackage{caption}
\usepackage{subcaption}

% Paket zur Erstellung von Zeichnungen
\usepackage{tikz}
\usetikzlibrary{positioning}

% Für SI Schreibweise
\usepackage{siunitx}

% Mathe
\usepackage{amsmath}

% Paket um Grafiken (JPG, PNG, PDF) einzubinden
\usepackage{graphicx}
\graphicspath{{pictures/}}

% Pakete für Tabellenlayout
\usepackage{booktabs}
\usepackage{tabularx}

% Paket für Syntaxhighlighting für inline code
\usepackage[minted]{tcolorbox}
\usemintedstyle{one_dark_pro_style} % good styles: default, vs, manny, xcode, lovelace
\tcbuselibrary{listingsutf8}
\newtcolorbox{codeblock}[1][]{%
  colback=boxBG,
  colframe=boxFrame,
  boxrule=0.4pt,        % Rahmenstärke
  arc=2mm,              % abgerundete Ecken
  outer arc=2mm,
  boxsep=2pt,           % Innenabstand
  left=2pt, right=2pt, top=2pt, bottom=2pt,
  listing only,
  minted options={fontsize=\small, breaklines, #1}
}

\tcbuselibrary{skins, breakable}
\newtcolorbox{imgbox}[1][]{%
  enhanced,
  colback=boxBG,
  colframe=boxFrame,
  boxrule=0.4pt,
  arc=2mm,
  outer arc=2mm,
  boxsep=3pt,
  left=3pt, right=3pt, top=3pt, bottom=3pt,
  #1
}

% Befehl für einzelne, farbige interne Links
\newcommand{\TeLi}[1]{%
  \begingroup
    \hypersetup{linkcolor=TeLicolor}%
    \ref{#1}%
  \endgroup
}

\newcommand{\hTeLi}[2]{%
  \begingroup
    \hypersetup{linkcolor=TeLicolor}%
    \hyperref[#1]{#2}%
  \endgroup
}

\newcommand{\aTeLi}[1]{%
  \begingroup
    \hypersetup{linkcolor=TeLicolor}%
    \autoref{#1}%
  \endgroup
}

% Paket für Zeilenabstand
\usepackage{setspace}

% Paket für Hyperlinks
\usepackage[
  colorlinks=true,
  linkcolor=black,      % For internal links (like table of contents)
  citecolor=black,      % For citations
  urlcolor=blue         % For URLs
]{hyperref}

% Paket für korrekte Anführungszeichen
\usepackage{csquotes}

% Paket für selbst definierte Kopf- und Fusszeilen
\usepackage{scrlayer-scrpage}

% Pakte für Zitate und Bibliografie
\usepackage[style=ieee, citestyle=numeric-comp]{biblatex}
\addbibresource{literature.bib}

% Paket zum Erzeugen von Platzhaltertext
\usepackage{blindtext}

% Schmale Seitenränder festlegen
\KOMAoption{DIV}{15}

% Variablen
\newcommand{\maTitle}{Autonom durch die Lüfte: Die Integration von Gestenerkennung in die Navigation eines Mini-Hubschraubers}
\newcommand{\chapOne}{Einleitung}
\newcommand{\chapTwo}{Theorie}
\newcommand{\chapThree}{Methode}
\newcommand{\chapFour}{Ergebnisse}
\newcommand{\chapFive}{Diskussion}

% Für Code im Fliesstext
\newcommand{\bodyCode}[1]{\texttt{\small #1}}

% Für Code der alleine steht
% Usage: \codefile[filename]{caption}{label}
\newcommand{\codefile}[3][]{%
    \begin{minipage}{\linewidth}
        \vspace{1ex}
        \captionsetup{type=listing,labelformat=empty}
        \caption*{\large{\textbf{#2}}}
        \label{#3}
        \vspace{-1ex}
        \begin{codeblock}[#1]
            \inputminted[
                fontsize=\small,
                breaklines=true,
                linenos=false
            ]{python}{code_snippets/#1}
        \end{codeblock}
    \end{minipage}
}

% Für Mathe im Fliesstext
\newcommand{\inlinemath}[1]{\(#1\)}

% Für Math
\newcommand{\cp}{\textcolor{plusGreen}{\texttt{+}}}
\newcommand{\cm}{\textcolor{minusRed}{\texttt{-}}}
\newcommand{\vect}[2]{\ensuremath{\begin{pmatrix} #1 \\ #2 \end{pmatrix}}}
% Für Tech-Wörter
\newcommand{\techWord}[1]{\textit{#1}}

% Text auf Titelseite festlegen
\subject{Maturaarbeit}
\title{\maTitle}
\author{Nelio Gautschi\vspace{1cm}\\
Betreut durch \\
Stefan Rothe}
\publishers{\includegraphics{logo}\vspace{1cm}\\
Gymnasium Kirchenfeld\\
Abteilung GH}

% selbst definierte Kopf- und Fusszeile
\lohead{Maturaarbeit}

%\rohead{\theauthor}
\cofoot{\thepage}

% Zeilenabstand festlegen
\singlespacing %\doublespacing \onehalfspacing

% Trennlinie für Kopfzeile
\KOMAoption{headsepline}{off}

% Trennlinie für Fusszeile
\KOMAoption{footsepline}{on}

% ANFANG DOKUMENT -----------------------------------
\begin{document}

% erstelle Titel
\maketitle

\chapter*{Danksagung}
\label{cha:ack}

\begingroup
\fontsize{12pt}{14pt}\selectfont

Ich möchte zuallererst Stefan Rothe für sein Vertrauen in mich und meine Ideen danken.  
Auch für sein Interesse und die Förderung meiner Begeisterung für das Programmieren bin ich ihm dankbar.

Ebenso danke ich Andreas Gautschi für sein professionelles Feedback, seine inspirierenden Anregungen sowie fürs Entfachen der Begeisterung am Programmieren.

Besonderer Dank gilt Inga Eickemeier Gautschi, die jedes Wort überprüft hat und sicherstellte, dass ich diese Arbeit im vorgegebenen zeitlichen Rahmen beenden konnte.

Herzlich danke ich Maxime Klinkert für ihre Rücksichtnahme, ihr Verständnis und ihre Unterstützung – trotz der vielen Nächte, in denen der Computer unser Schlafzimmer beleuchtete.

Schliesslich gilt mein Dank Rufus Spyra für seine ermunternden Worte und Ideen in nächtlichen Krisensituationen.

\endgroup

\tableofcontents
\clearpage

\begingroup
    \let\clearpage\relax
    \listoffigures
    \listoftables
\endgroup

\chapter{\chapOne}
\label{cha:chapter1} % Label for hyperlink

% Start font-size
\begingroup
\fontsize{12pt}{14pt}\selectfont

\section{Einführung ins Thema}
Gestenerkennung und Gestensteuerung haben sich in den letzten Jahren von Nischenanwendungen im Gaming-Bereich wie beispielsweise der Kinect-Plattform~\cite{Wiki:Kinect} zu vielseitigen Steuerungsmöglichkeiten für virtuelle und reale Umgebungen entwickelt~\cite{RG:GestureRecognition}.
Ihre Wurzeln liegen teilweise in der Analyse der Gebärdensprache, reichen jedoch letztlich bis zu den Anfängen menschlicher Kommunikation zurück.\cite{Wiki:Gestenerkennung}\cite{RG:Gesten}
Gesten stellen einen festen und wesentlichen Bestandteil der nonverbalen Verständigung dar und bilden somit eine natürliche Grundlage für intuitive Interaktionsformen zwischen Mensch und Maschine.\cite[10]{Hobmair:Psy}

Parallel dazu gewinnt die autonome Steuerung technischer Systeme zunehmend an Bedeutung.
Fahrzeuge, die mit modernen Fahrassistenzsystemen ausgestattet sind, können heute vom Spurhalten bis zum automatischen Parkieren fast alles selbstständig ausführen.
Doch nicht nur im Strassenverkehr, auch in der Luft- und Schifffahrt werden Prozesse automatisiert, um Effizienz und Sicherheit zu erhöhen und menschliche Fehler zu minimieren.
Trotz dieser Entwicklungen bleibt der Mensch ein zentraler Bestandteil des Kontrollsystems: Er kann jederzeit in den automatisierten Ablauf eingreifen und die Steuerung übernehmen.\cite{Wiki:aupi}

Noch rasanter schreitet die Entwicklung im Bereich der virtuellen und erweiterten Realität (VR und AR) voran~\cite{SD:VR}.
Hier ermöglichen innovative Interaktionskonzepte eine immer natürlichere, präzisere und intuitivere Steuerung digitaler Umgebungen.
Ein beeindruckendes Bespiel ist dabei die \textit{Apple Vision Pro}, eine VR-Brille, die im Jahr 2023 auf den Märkten erschienen ist.
Das Gerät verfolgt die Bewegung der Augen und verwendet diese als Gestensteuerung und erlaubt es durch Zusammenführen von Daumen und Zeigefinger Objekte auf dem virtuellen Bildschirm auszuwählen.\cite{apl:vision}

Auch in der Robotik eröffnen Gestensteuerungssysteme neue Perspektiven: Sie erlauben eine direkte, beinahe natürliche Kommunikation mit semi-autonomen oder autonomen Geräten, etwa bei der Navigation von Drohnen oder mobilen Robotern.
So können beispielsweise Bergungsarbeiten oder Minenräumungen künftig aus sicherer Entfernung durchgeführt werden, ohne dabei Menschen unnötigen Gefahren auszusetzen.

\section{Zielsetzung der Arbeit}
Das Ziel dieser Arbeit ist es, die Konzepte der Gestenerkennung sowie deren Einsatzmöglichkeiten zur Steuerung virtueller Systeme zu veranschaulichen.
Als Grundlage dient dabei ein ähnliches Projekt, welches ein neurales Netzwerk zur Erkennung von Handpositionen implementiert, um eine Steuerungsalternative zu traditionellen Kontrollern zu bieten.
Die Position der Hand wird von einer Kamera erfasst und von einem Computer ausgewertet.
Dieser sendet Steuerbefehle anhand der berechneten Daten an einer Renndrohne.\cite{arxiv:OmniRace}

Um den Aufwand in einem realistischen Rahmen zu halten, wird in dieser Arbeit nicht auf eine direkte Erkennung der Fingerpositionen gesetzt.
Stattdessen kommen vier ArUco-Marker zum Einsatz, die auf der Steuerhand befestigt werden.
Dies soll eine präzise Erkennung der Handausrichtung ermöglichen, indem Flächen, Positionen und Distanzen der Marker berechnet und verglichen werden.
Anschliessend wird mit den in \hTeLi{sec:cf}{Kapiteln~\ref{sec:cf}} und \hTeLi{sec:crpa}{\ref{sec:crpa}} beschriebenen Hardware-Komponenten eine Drohne gesteuert.

In diesem Projekt werden zwei Ansätze getestet, die sich in der Platzierung der Marker unterscheiden.
Die Anordnung der Marker soll die Bedeutsamkeit veranschaulichen, die Finger beim Gebrauch von Gesten haben.
Der erste Ansatz sieht die Marker in einem Rechteck angeordnet vor, wobei ausschliesslich die Handfläche als Bereich für die Marker dient.
Der zweite Versuch setzt hingegen drei von vier Markern auf den Fingerspitzen vorraus.
Der vierte Marker wird auf dem Handgelenk platziert und markiert für die Kamera den Referenzpunkt der Hand.

\subsection{Fragestellung}
Wie kann ein System zur gestenbasierten Steuerung einer Nano-Drohne entwickelt werden, das mithilfe von ArUco-Markern Handgesten zuverlässig erkennt, diese in präzise Steuerbefehle umwandelt und die Drohne per  Funkverbindung sicher und reaktionsschnell steuert?

\subsection{Abgrenzung (ausgeschlossene Faktoren)}
Im Rahmen dieses Projekts werden externe Einflussfaktoren bewusst ausgeklammert bzw. nicht im Detail untersucht.
Dazu gehören insbesondere:
\begin{itemize}
  \item \textbf{Licht- und Belichtungsverhältnisse:} Es wird von konstanten, guten Lichtverhältnissen ausgegangen (z.B. gleichmässige Raumbeleuchtung ohne starke Schatten oder Überbelichtung).
  \item \textbf{Wind- und Luftströmungen:} Das System wird ausschliesslich in einer windstillen Indoor-Umgebung getestet.
  \item \textbf{Funkstörungen:} Potenzielle Interferenzen durch andere Funkgeräte oder Netzwerke werden nicht berücksichtigt.
  \item \textbf{Komplexe Handgesten:} Es wird davon ausgegangen, dass die ArUco-Tags jederzeit für die Kamera sichtbar bleiben.
\end{itemize}

\subsection{Hypothese}
Die Nutzung von ArUco-Markern zur Gestenerkennung ermöglicht eine robuste Gestensteuerung der Drohne, wobei Positions- und Bewegungsdaten der Marker ausreichend präzise sind, um alle grundlegenden Steuerbefehle (z.B. Steigen, Sinken, Drehen) sicher und ohne Fehlinterpretationen auszuführen.

% End font-size
\endgroup
\chapter{\chapTwo}
\label{sec:kapitel2} % Label for hyperlink

% Start font-size
\begingroup
\fontsize{12pt}{14pt}\selectfont

\section{OpenCV}
\label{sec:opencv} % Label for hyperlink

\section{ArUco Marker}
Ein ArUco Marker ist ein quadratischer Referenzmarker, der in räumlichen Messungen zum Einsatz kommt. Referenzmarker (auch Passermarken) werden ursprünglich in automatisierten Fertigungsverfahren von elektronischen Bauelementen wie beispielsweise Hauptplatinen benutzt. Sie dienen als optische Referenzpunkte für die Maschinen, die im Anfertigungsprozess involviert sind, um Präzision zu erhöhen und Kurzschlüsse zu reduzieren \cite{Wiki:Passermarke}. 
Es gibt Referenzmarker in verschiedenen Grössen, Farbschemen und Formen, abhängig von ihrem Einsatzgebiet. Für diese Arbeit werden aus folgenden Gründen quadratische benutzt:

\begin{itemize}
  \item Die Erkennung der Marker erfolgt schnell und effizient, was bei Echtzeitsteuerung vorteilhaft ist.
  \item Quadratische Marker werden häufig gebraucht und sind daher ausführlicher dokumentiert als andere.
  \item Sie sind vergleichsweise einfach einzusetzen.
\end{itemize}

ArUco-basierte Marker zählen zu den verlässlichsten, vorallem seit \hyperref[sec:opencv]{OpenCV} ein Submodul für deren Implementierung hinzugefügt hat \cite{IJ:fiducial}.

\begin{figure}[h!]
    \centering
    \includegraphics[width=0.3\textwidth]{aruco.pdf}
    \caption{ArUco-Marker mit ID 0. Übernommen von \cite{chev:arucogen}}
    \label{fig:marker0}
\end{figure}

% End font-size
\endgroup
ar\chapter{\chapThree}
\label{cha:chapter3} % Label for hyperlink

% Start font-size
\begingroup
\fontsize{12pt}{14pt}\selectfont

In diesem Kapitel wird beschrieben, wie die im Kapitel 2 vorgestellten theoretischen Grundlagen praktisch umgesetzt werden.
Ziel ist es, eine Gestensteuerung der Crazyflie 2.0 mithilfe von ArUco-Markern zu realisieren.
Um dieses Ziel zu erreichen, wurde eine Methode entwickelt, die sowohl die Erkennung und Analyse von Handbewegungen als auch die Steuerung der Drohne umfasst.

\section{Vorgehensweise}

Diese Arbeit umfasst mehrere Prozessschritte, die von der Erfassung visueller Daten über deren mathematische Verarbeitung bis hin zur Umsetzung in Steuerbefehle für die Drohne reichen.
Dazu werden folgende Schritte durchgeführt:

\begin{enumerate}
    \item \textbf{Erfassen der Handbewegung:}
    \begin{itemize}
        \item Kameraerfassung der ArUco-Marker auf der Hand.
        \item Verarbeitung der Bilder mit OpenCV.
        \item Extraktion der Markerpositionen und Eckpunkte.
    \end{itemize}
    \item \textbf{Datenverarbeitung und Interpretation:}
    \begin{itemize}
        \item Flächenberechnung jedes Markers mittels Gaußscher Trapezformel.
        \item Bestimmung der Ausrichtung der Handfläche.
    \end{itemize}
    \item \textbf{Steuerung der Crazyflie:}
    \begin{itemize}
        \item Übersetzung der Handausrichtung in Steuerargumente.
        \item Verwendung dieser Argumente in den CFLib-Funktionen.
        \item Übertragung der Befehle über das Crazyradio PA via CRTP.
    \end{itemize}
\end{enumerate}

\section{Erfassen der Handbewegung}

Zu Beginn muss die OpenCV-Bibliothek in das Python-Skript importiert werden, da sie die grundlegenden Funktionen für die Bildverarbeitung bereitstellt.
Für die Erfassung der Handbewegung ist der Zugriff auf die Kamera erforderlich.
Dies erfolgt über die Klasse \bodyCode{VideoCapture()}, die eine Verbindung zur Kamera herstellt und kontinuierlichen Zugriff auf deren Bilddaten ermöglicht.
Ob die Kamera erfolgreich geöffnet wurde oder beispielsweise durch ein anderes Programm blockiert ist, kann mit der integrierten Funktion \bodyCode{isOpened()} überprüft werden.

Um sicherzustellen, dass tatsächlich Bildinformationen empfangen werden, muss in einer Schleife die \bodyCode{read()}-Funktion aufgerufen werden.
Diese gibt zwei Werte zurück:

\begin{itemize}
    \item Der erste Wert gibt an, ob ein Bild erfolgreich empfangen wurde\footnotemark{}.
    \footnotetext{Im \hTeLi{sni:det}{Codeausschnitt} als \bodyCode{success} ersichtlich.}
    \item Der zweite Wert enthält das aktuelle Kamerabild in Form eines NumPy-Arrays\footnotemark{}.
    \footnotetext{Im \hTeLi{sni:det}{Codeausschnitt} als \bodyCode{frame} ersichtlich.}
\end{itemize}

Wenn kein Bild empfangen werden konnte, wird das Programm beendet.
Anschliessend wird das Bild in Graustufen konvertiert, um die spätere Erkennung der ArUco-Marker zu erleichtern.
Die Spiegelung entlang der vertikalen Achse sorgt dafür, dass das angezeigte Bild der realen Ausrichtung der Hand entspricht.

Damit im aufgenommenen Bild nach ArUco-Markern gesucht werden kann, muss zunächst die passende ArUco-Bibliothek definiert werden.
Das geschieht einmalig mit dem Befehl

\bodyCode{cv2.aruco.getPredefinedDictionary(cv2.aruco.DICT\_4X4\_50)}.

Wie bereits in \hTeLi{sub:fw}{Kapitel~\ref{sub:fw}} erwähnt, werden im Rahmen dieser Arbeit Marker aus der \bodyCode{4X4\_50}-Bibliothek verwendet.
Mit \bodyCode{DetectorParameters()} wird eine Instanz der im ArUco-Modul enthaltenen Parameterklasse erstellt.
Diese Klasse definiert unter anderem die internen Abläufe der Markererkennung wie beispielsweise die Kantenerkennung oder die Filterung von Fehlkandidaten.

Das Parameterobjekt und die festgelegte Bibliothek werden anschliessend als Argumente an die Funktion \bodyCode{ArucoDetector()} übergeben, welche den eigentlichen Erkennungsablauf initialisiert.
Die Detektion erfolgt durch den Aufruf von \bodyCode{detectMarkers()}, der das aktuelle Bild als Argument entgegennimmt.
Als Rückgabewert erhält man eine geordnete Sammlung (ein sogenanntes Tupel) dreier Elemente.

\begin{itemize}
    \item Das erste Element enthält die Eckkoordinaten der erkannten Marker in Form von NumPy-Arrays.
    \item Das zweite Element enthält die zugehörigen Marker-IDs.
    \item Das dritte Element umfasst Marker-Kandidaten\footnotemark{}, die zwar als potenzielle Marker erkannt, aber nach der Validierung verworfen wurden.
    \footnotetext{Im \hTeLi{sni:det}{Codeausschnitt} als \bodyCode{rejected} ersichtlich.}
\end{itemize}

\codefile[detection.py]{Codeausschnitt: Initialisierung der Kamera und Erfassung der ArUco-Marker}{sni:det}

\subsection{Potenzielle Probleme}
Bei der Erkennung der ArUco-Marker wird davon ausgegangen, dass verschiedene Faktoren die Genauigkeit und Stabilität der Resultate beeinflussen könnten.

Ein wesentlicher Einflussfaktor dürften die Lichtverhältnisse sein.
Starke Reflexionen, Schatten oder ungleichmässige Beleuchtung könnten dazu führen, dass die Kanten der Marker nicht eindeutig erkannt werden.
Dies würde sich insbesondere auf die Kantenerkennung und damit auf die Validierung der Marker auswirken.

Auch die Grösse der Marker sowie der Abstand zur Kamera könnten eine Rolle spielen.
Befinden sich die Marker zu weit entfernt oder werden sie in zu kleiner Auflösung aufgenommen, wären die schwarzen und weissen Bereiche nicht mehr klar voneinander unterscheidbar, was die Erkennung erschweren könnte.
Ebenso wäre es denkbar, dass eine teilweise Verdeckung oder eine starke Neigung der Hand dazu führt, dass Marker nicht erkannt oder falsch identifiziert würden.

Darüber hinaus könnte Bewegung während der Aufnahme zu unscharfen Bildern führen.
Dies träfe insbesondere dann zu, wenn die Framerate der Kamera zu niedrig oder die Bewegung der Hand zu schnell wäre.

\section{Datenverarbeitung und Interpretation}

\section{Steuerung der Crazyflie}
\label{sec:cf_co}

% End font-size
\endgroup
\chapter{\chapFour}
\label{cha:chapter4} % Label for hyperlink

% Start font-size
\begingroup
\fontsize{12pt}{14pt}\selectfont

In diesem Kapitel werden die Ergebnisse der durchgeführten Tests dargestellt.
Untersucht wurden sowohl die Genauigkeit der Gestenerkennung als auch die Reaktionsfähigkeit und Stabilität der Drohnensteuerung.
Die Tests wurden unter verschiedenen Bedingungen durchgeführt, um die Zuverlässigkeit und das Echtzeitverhalten des Gesamtsystems zu bewerten.
Als Evaluationskriterien dienten unter anderem Erkennungsrate, Reaktionszeit und Genauigkeit der Steuerbefehle.
Da Steuerungssysteme intuitiv wirken sollten, wurden vier Personen gebeten, das System auszuprobieren und Feedback zu geben.
Intuition ist somit ebenfalls ein Kriterium.

\section{Einzelergebnisse}

Die im Rahmen der Arbeit erhobenen Resultate wurden in drei Kategorien unterteilt: \hTeLi{sub:geRec}{Gestenerkennung}, \hTeLi{sec:drco}{Drohnensteuerung} und \hTeLi{sec:whsy}{Gesamtsystem}.
Diese Einteilung ermöglicht eine Trennung zwischen der Bildverarbeitung, der Steuerlogik und dem praktischen Zusammenspiel beider Komponenten.
So lassen sich Stärken und Schwächen der einzelnen Systemteile gezielt analysieren und bewerten.

\subsection{Gestenerkennung}
\label{sub:geRec}

Um die Leistung der Gestenerkennung zu testen, wurde eine Funktion in das Skript implementiert, die die Ecken der erkannten Marker auf dem angezeigten Videobild rot darstellt.
So kann visuell bestimmt werden, ob ein Marker im jetzigen Moment erkannt wird.
Es ist festzuhalten, dass die Bildaufnahmegeschwindigkeit Einfluss auf die folgenden Resultate hat.\footnotemark{}
\footnotetext{Es wurden \textbf{nicht} mehrere Kameras ausprobiert.}

Zunächst wurde die Detektion von bis zu 35 Markern gleichzeitig mit der eines einzelnen Markers verglichen.
So viele Marker auf einmal zu erkennen, ist für das Skript grundsätzlich kein Problem, findet jedoch Bewegung statt, so nimmt die Genauigkeit um bis zu \SI{25}{\percent} ab.
Man merkt auch einen Unterschied, wenn man vier anstatt einen Marker zu erkennen hat.
Da alle vier Marker mehrmals gleichzeitig pro Sekunde erkannt werden, genügt die Genauigkeit für die Drohnensteuerung.
Für diesen Test wurden die ArUco-Tags aus Platzgründen nicht auf die Hand geklebt.

Im nächsten Schritt wurden die tatsächlichen Anordnungen der Marker der jeweiligen Methode verwendet.
Anordnung~1 steht dabei für die Anordnung der Methode~1, Anordnung~2 entsprechend für die Anordnung der Methode~2.
Es wurden zwei Versuche durchgeführt, bei denen die Erkennung der Tags unter Winkeln geprüft wurde.
Das Resultat der Anordnung~1 zeigt erstmals, dass Unterschiede in der Anwendung anzutreffen sind.
Der sogenannte Thenarmuskel, der Muskel des Daumens, verursacht bei Tag mit ID 3 eine Untergrundswölbung, welche die Oberfläche des Markers ungewollt verzerrt.
Das verkleinert den Erkennungsbereich-Winkel der Methode~1 erheblich.
Anordnung~2 schneidet hierbei besser ab.
Sie wird effizient erkannt bis zu circa 35° Anwinkelung\footnotemark{}.
\footnotetext{\SI{45}{\degree} entsprechen einem Seitenbild der Hand, wo die Handfläche nicht mehr zu sehen ist.}

Als nächstes wird der ideale Abstand zwischen der Kamera und der Hand untersucht.
Für beide Methoden wird die beste Reliabilität der Marker-Erkennung bei einer Entfernung von \SI{20}{\centi\meter} erreicht.

Zum Schluss wurden noch die Lichtverhältnisse analysiert.
Die Marker-Erkennung kann grundsätzlich ohne Umgebungslicht durchgeführt werden, weil das vom Bildschirm ausgestrahlte Licht ausreicht.
Tatsächlich funktioniert der Prozess am besten, wenn es dunkler ist.
Befinden sich Lichtquellen im Raum, dürfen sie nicht direkt in die Kameralinse strahlen.
Wird dies nicht beachtet, ist eine Überbelichtung des erfassten Bildes und somit nur eine schlechte bis gar keine Marker-Erkennung möglich.

\subsection{Drohnensteuerung}
\label{sec:drco}

Das Hauptproblem der Crazyflie~2.0 wird ersichtlich, wenn Hindernisse überflogen werden.
Der Höhensensor erkennt einen abrupten Wechsel in der Messung der Entfernung zum Untergrund und interpretiert dies als ungewollten Höhenverlust oder -gewinn.
Die Drohne versucht den Höhenunterschied auszugleichen und fliegt in unvorhersehbare Richtungen.
Aus diesem Grund wurde für diese Arbeit eine Umgebung ohne Hindernisse und mit mattem Boden gewählt.

Ein weiteres Problem stellen andere Geräte dar, die Datenverkehr auf der \SI{2.4}{\giga\hertz} Frequenz betreiben.
Erfasst die Drohne ein gewisses Mass an Paketverlust, bricht sie die Verbindung zum Crazyradio ab.
Um das Auftreten von ungewolltem Kommunikationsabbruch zur Drohne zu minimieren und die Stabilität der Verbindung zu fördern, wird eine Datenrate von \SI{2}{\mega\bit\per\second} anstatt der Standardeinstellung \SI{250}{\kilo\bit\per\second} verwendet.

Die Drohnensteuerung wurde zunächst ohne Eingabesteuerung geprüft, um die Auswirkung der markerbasierten Steuerung zu bestimmen.
Dank des Flow-Deck~V2 werden Distanz- und Geschwindigkeitsbefehle präzise ausgeführt, mit einer Abweichung im Zentimeterbereich.

Eine verzögerte Reaktionszeit konnte nicht festgestellt werden, da sie unterhalb der für diese Arbeit relevanten Zeitspanne liegt.

\subsection{Gesamtsystem}
\label{sec:whsy}

Das Gesamtsystem wird mithilfe eines Zeitrennens getestet.
Dazu wird den Wänden eines viereckigen Raumes ohne Hindernisse entlang geflogen.
Insgesamt müssen vier \SI{90}{\degree} Linkskurven und \si{36} Meter Länge geflogen werden, um am Startpunkt, der auch das Ziel ist, wieder anzukommen.

Bei Verwendung der ersten Methode fliegt die Drohne für \si{115} Sekunden, bis sie das Ziel erreicht.
Dabei sind nicht nur Links- und Rechtsbewegungen problematisch, sondern auch die einfache Vorwärtsbewegung.
Neben der Ungenauigkeit der Richtungsbestimmung ist bei Methode~1 auch eine höhere Sensibilität auf Handbewegungen zu spüren, obwohl sie dieselben \textit{Deadzones} verwenden.
Diese Sensibilität ist vermutlich auf die Distanz zwischen den Markern zurückzuführen.
Die grössere Spannweite der Marker in Methode~2 ermöglicht präzisere Richtungsänderungen.

Beim Gebrauch der zweiten Methode endet der Flug bereits nach \si{90} Sekunden. 
Bei der gewählten Geschwindigkeit von \SI{0.5}{\meter\per\second} und ohne Beschleunigungsfunktionalität sind das \si{18} Sekunden mehr als die theoretisch mögliche Bestzeit.\footnotemark{}
\footnotetext{Beide Methoden verwendeten dieselben Geschwindigkeitswerte.}
Nach drei weiteren Runden pro Methode lassen sich die durchschnittlichen Rennzeiten berechnen:

\begin{itemize}
    \item \textbf{Methode~1:} \si{117} Sekunden
    \item \textbf{Methode~2:} \si{73} Sekunden
\end{itemize}

Zum Schluss wurde das Feedback der vier Testpersonen ausgewertet.
Als Antwortmöglichkeiten wurde \enquote{intuitiv}, \enquote{eher intuitiv} und \enquote{nicht intuitiv} vorgegeben.
In folgender Tabelle sind die Antworten aufgeführt:

\begin{table}[H]
    \centering
    \begin{tabular}{l|l}
        \textbf{Testende Person} & \textbf{Einschätzung nach Verwendung} \\ \hline
        Nummer 1 & eher intuitiv \\
        Nummer 2 & nicht intuitiv \\
        Nummer 3 & intuitiv \\
        Nummer 4 & eher intuitiv \\
        \hline
    \end{tabular}
    \caption{Feedback der Testpersonen}
        \label{tab:fdbk}
\end{table}

\section{Zusammenfassung der Ergebnisse}
Die durchgeführten Tests zeigen, dass das System grundsätzlich zuverlässig arbeitet und eine markerbasierte Gestensteuerung der Crazyflie 2.0 möglich ist.
Die Gestenerkennung erwies sich als stabil, solange die Marker nicht stark geneigt oder überbelichtet waren.
Die beste Erkennung wurde bei einer Entfernung von rund \SI{20}{\centi\meter} und schwacher Umgebungsbeleuchtung erreicht.
Die Drohnensteuerung reagierte präzise auf übertragene Befehle, wobei durch Funkstörungen oder unebene Untergründe gelegentlich Stabilitätsprobleme auftraten.
Eine messbare Verzögerung der Reaktionszeit lag nicht vor.
Im kombinierten Testlauf schnitt Methode~2 deutlich besser ab.
Sie erlaubte präzisere Richtungsänderungen und eine kürzere Gesamtlaufzeit von durchschnittlich \si{73} Sekunden gegenüber \si{117} Sekunden bei Methode~1.
Das Feedback der Testpersonen bestätigt diesen Eindruck:
Drei von vier Personen empfanden die Steuerung als intuitiv oder eher intuitiv.

Insgesamt zeigt sich, dass die zweite Methode sowohl technisch als auch in der subjektiven Bewertung überlegen ist und sich für eine gestenbasierte Steuerung der Crazyflie besser eignet.

% End font-size
\endgroup
\chapter{\chapFive}
\label{cha:chapter5} % Label for hyperlink

% Start font-size
\begingroup
\fontsize{12pt}{14pt}\selectfont

\blindtext

% End font-size
\endgroup


\huge\textbf{Hilfsmittelverzeichnis}

\newcommand{\tabTitle}[1]{\textbf{\textit{\large #1}}}
\renewcommand{\arraystretch}{1.5} % 1.0 = Standard, 1.5 = 50% höher

\begin{table}[H]
    \centering
    \begin{tabularx}{\textwidth}{|l|X|X|X|}
        \hline
        \tabTitle{Hilfsmittel} & \tabTitle{Einsatzform} & \tabTitle{Bereich} & \tabTitle{Bemerkung} \\
        \hline

        \textbf{ChatGPT (OpenAI)} & Erstellen der LaTeX-Dateien & Gesamte Arbeit & Formatieren des Fliesstextes \\
         & Textüberarbeitung & Gesamte Arbeit & Syntax, Wortwiederholungen \\ 
         & Gedankenstütze & Kapitel \ref{sub:gauTrap} & Umformung Trapezformel zu Shoelace formula \cite{oai:chatgpt} \\
        \hline
    \end{tabularx}
        \label{tab:tools}
\end{table}

\renewcommand{\arraystretch}{1.0}

\printbibliography

\chapter*{Anhang}

\begin{center}
    \includegraphics[width=0.5\textwidth]{github_repo.pdf}
    \vspace{0.5cm}
    \textbf{Abbildung:} QR-Code zu meinem Github-Repository \cite{git:repo}.
\end{center}

\end{document}
% ENDE DOKUMENT -----------------------------------