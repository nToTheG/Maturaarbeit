\chapter{\chapThree}
\label{cha:chapter3} % Label for hyperlink

% Start font-size
\begingroup
\fontsize{12pt}{14pt}\selectfont

In diesem Kapitel wird beschrieben, wie die im Kapitel 2 vorgestellten theoretischen Grundlagen praktisch umgesetzt werden.
Ziel ist es, eine Gestensteuerung der Crazyflie 2.0 mithilfe von ArUco-Markern zu realisieren.
Um dieses Ziel zu erreichen, wurde eine Methode entwickelt, die sowohl die Erkennung und Analyse von Handbewegungen als auch die Steuerung der Drohne umfasst.

\section{Vorgehensweise}

Diese Arbeit umfasst mehrere Prozessschritte, die von der Erfassung visueller Daten über deren mathematische Verarbeitung bis hin zur Umsetzung in Steuerbefehle für die Drohne reichen.
Dazu werden folgende Schritte durchgeführt:

\begin{enumerate}
    \item \textbf{Erfassen der Handbewegung:}
    \begin{itemize}
        \item Kameraerfassung der ArUco-Marker auf der Hand.
        \item Verarbeitung der Bilder mit OpenCV.
        \item Extraktion der Markerpositionen und Eckpunkte.
    \end{itemize}
    \item \textbf{Datenverarbeitung und Interpretation:}
    \begin{itemize}
        \item Flächenberechnung jedes Markers mittels Gaußscher Trapezformel.
        \item Bestimmung der Ausrichtung der Handfläche.
    \end{itemize}
    \item \textbf{Steuerung der Crazyflie:}
    \begin{itemize}
        \item Übersetzung der Handausrichtung in Steuerargumente.
        \item Verwendung dieser Argumente in den CFLib-Funktionen.
        \item Übertragung der Befehle über das Crazyradio PA via CRTP.
    \end{itemize}
\end{enumerate}

\section{Erfassen der Handbewegung}
\blindtext

% End font-size
\endgroup