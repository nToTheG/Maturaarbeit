\chapter{\chapTwo}
\label{sec:kapitel2} % Label for hyperlink

% Start font-size
\begingroup
\fontsize{12pt}{14pt}\selectfont

\section{OpenCV}
\label{sec:opencv} % Label for hyperlink

\section{ArUco Marker}
Ein ArUco Marker ist ein quadratischer Referenzmarker, der in räumlichen Messungen zum Einsatz kommt. Referenzmarker (auch Passermarken) werden ursprünglich in automatisierten Fertigungsverfahren von elektronischen Bauelementen wie beispielsweise Hauptplatinen benutzt. Sie dienen als optische Referenzpunkte für die Maschinen, die im Anfertigungsprozess involviert sind, um Präzision zu erhöhen und Kurzschlüsse zu reduzieren \cite{Wiki:Passermarke}. 
Es gibt Referenzmarker in verschiedenen Grössen, Farbschemen und Formen, abhängig von ihrem Einsatzgebiet. Für diese Arbeit werden aus folgenden Gründen quadratische benutzt:

\begin{itemize}
  \item Die Erkennung der Marker erfolgt schnell und effizient, was bei Echtzeitsteuerung vorteilhaft ist.
  \item Quadratische Marker werden häufig gebraucht und sind daher ausführlicher dokumentiert als andere.
  \item Sie sind vergleichsweise einfach einzusetzen.
\end{itemize}

ArUco-basierte Marker zählen zu den verlässlichsten, vorallem seit \hyperref[sec:opencv]{OpenCV} ein Submodul für deren Implementierung hinzugefügt hat \cite{IJ:fiducial}.

\begin{figure}[h!]
    \centering
    \includegraphics[width=0.3\textwidth]{aruco.pdf}
    \caption{ArUco-Marker mit ID 0. Übernommen von \cite{chev:arucogen}}
    \label{fig:marker0}
\end{figure}

% End font-size
\endgroup