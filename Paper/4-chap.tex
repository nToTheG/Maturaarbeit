\chapter{\chapFour}
\label{cha:chapter4} % Label for hyperlink

% Start font-size
\begingroup
\fontsize{12pt}{14pt}\selectfont

In diesem Kapitel werden die Ergebnisse der durchgeführten Tests dargestellt.
Untersucht wurden sowohl die Genauigkeit der Gestenerkennung als auch die Reaktionsfähigkeit und Stabilität der Drohnensteuerung.
Die Tests wurden unter verschiedenen Bedingungen durchgeführt, um die Zuverlässigkeit und das Echtzeitverhalten des Gesamtsystems zu bewerten.
Als Evaluationskriterien dienten unter anderem Erkennungsrate, Reaktionszeit und Genauigkeit der Steuerbefehle.
Da Steuerungssysteme intuitiv wirken sollten, wurden vier Personen gebeten, das System auszuprobieren und Feedback zu geben.
Intuition ist somit ebenfalls ein Kriterium.

\section{Einzelergebnisse}

Die im Rahmen der Arbeit erhobenen Resultate wurden in drei Kategorien unterteilt: \hTeLi{sub:geRec}{Gestenerkennung}, \hTeLi{sec:drco}{Drohnensteuerung} und \hTeLi{sec:whsy}{Gesamtsystem}.
Diese Einteilung ermöglicht eine Trennung zwischen der Bildverarbeitung, der Steuerlogik und dem praktischen Zusammenspiel beider Komponenten.
So lassen sich Stärken und Schwächen der einzelnen Systemteile gezielt analysieren und bewerten.

\subsection{Gestenerkennung}
\label{sub:geRec}

Um die Leistung der Gestenerkennung zu testen, wurde eine Funktion in das Skript implementiert, die die Ecken der erkannten Marker auf dem angezeigten Videobild rot darstellt (siehe !!TODO!!).
So kann visuell bestimmt werden, ob ein Marker im jetzigen Moment erkannt wird.
Es ist festzuhalten, dass die Bildaufnahmegeschwindigkeit Einfluss auf die folgenden Resultate hat.\footnotemark{}
\footnotetext{Es wurden \textbf{nicht} mehrere Kameras ausprobiert.}

Zunächst wurde die Detektion von bis zu 35 Markern gleichzeitig mit der eines einzelnen Markers verglichen.
So viele Marker auf einmal zu erkennen ist für das Skript grundsätzlich kein Problem, findet jedoch Bewegung statt, so nimmt die Genauigkeit um bis zu \SI{25}{\percent} ab.
Man merkt auch einen Unterschied, wenn man vier anstatt einem Marker zu erkennen hat.
Da alle vier Marker mehrmals gleichzeitig pro Sekunde erkannt werden, genügt die Genauigkeit für die Drohnensteuerung.
Für diesen Test wurden nicht die ArUco-Tags natürlich aus Platzgründen nicht auf die Hand geklebt.

Im nächsten Schritt wurden die tatsächlichen Anordnungen der Marker der jeweiligen Methode verwendet.
Anordnung~1 steht dabei für die Anordnung der Methode~1, Anordnung~2 entsprechend für die Anordnung der Methode~2.
Es wurden zwei Versuche durchgeführt, bei denen die Erkennung der Tags unter Winkeln geprüft wurde.
Das Resultat der Anordnung~1 zeigt erstmals, dass Unterschiede in der Anwendung anzutreffen sind.
Der sogenannte Thenarmuskel, der Muskel des Daumens, verursacht bei Tag mit ID 3 eine Untergrundswölbung, welche die Oberfläche des Markers ungewollt verzerrt.
Das verkleinert den Erkennungswinkel-Bereich der Methode~1 erheblich.
Anordnung~2 schneidet hierbei besser ab.
Sie wird effizient erkannt bis zu circa 35° Anwinkelung\footnotemark{}.
\footnotetext{\SI{45}{\degree} entsprechen einem Seitenbild der Hand, wo die Handfläche nicht mehr zu sehen ist.}

Als nächstes wird der ideale Abstand zwischen der Kamera und der Hand untersucht.
Für beide Methoden wird die beste Reliabilität der Marker-Erkennung bei einer Entfernung von \SI{20}{\centi\meter} erreicht.

Zum Schluss wurden noch die Lichtverhältnisse analysiert.
Die Marker-Erkennung kann grundsätzlich ohne Umgebungslicht durchgeführt werden, weil das vom Bildschirm ausgestrahlte Licht ausreicht.
Tatsächlich funktioniert der Prozess am besten, wenn es dunkler ist.
Befinden sich Lichtquellen im Raum, dürfen sie nicht direkt in die Kameralinse strahlen.
Wird dies nicht beachtet, ist eine Überbelichtung des erfassten Bildes und somit nur schlechte bis gar keine Marker-Erkennung möglich.

\subsection{Drohnensteuerung}
\label{sec:drco}

Das Hauptproblem der Crazyflie~2.0 wird ersichtlich, wenn Hindernisse überflogen werden.
Der Höhensensor erkennt einen abrupten Wechsel in der Messung der Entfernung zum Untergrund und interpretiert dies als ungewollten Höhenverlust oder -gewinn.
Die Drohne versucht den Höhenunterschied auszugleichen und fliegt in unvorhersehbare Richtungen.
Aus diesem Grund wurde für diese Arbeit eine Umgebung ohne Hindernisse und mit mattem Boden gewählt.

Ein weiteres Problem stellen andere Geräte dar, die Datenverkehr auf der \SI{2.4}{\giga\hertz} Frequenz betreiben.
Erfasst die Drohne ein gewisses Mass an Paketverlust, bricht sie die Verbindung zum Crazyradio ab.
Um das Auftreten von ungewolltem Kommunikationsabbruch zur Drohne zu minimieren und die Stabilität der Verbindung zu fördern, wird eine Datenrate von \SI{2}{\mega\bit\per\second} anstatt der Standardeinstellung \SI{250}{\kilo\bit\per\second} verwendet.

Die Drohnensteuerung wurde zunächst ohne Eingabesteuerung geprüft, um die Auswirkung der markerbasierten Steuerung zu bestimmen.
Dank des Flow-Deck~V2 werden Distanz- und Geschwindigkeitsbefehle präzise ausgeführt, mit einer Abweichung im Zentimeterbereich.

Eine verzögerte Reaktionszeit konnte nicht festgestellt werden, da sie unterhalb der für diese Arbeit relevanten Zeitspanne liegt.

\subsection{Gesamtsystem}
\label{sec:whsy}

Das Gesamtsystem wird mithilfe eines Zeitrennens getestet.
Dazu wird den Wänden eines viereckigen Raumes ohne Hindernisse entlang geflogen.
Insgesamt müssen vier \SI{90}{\degree} Linkskurven und \si{36} Meter Länge geflogen werden, um am Startpunkt, der auch das Ziel ist, wieder anzukommen.

Bei Verwendung der ersten Methode fliegt die Drohne für \si{115} Sekunden, bis sie das Ziel erreicht.
Dabei sind nicht nur Links- und Rechtsbewegungen problematisch, sondern auch die einfache Vorwärtsbewegung.
Neben der Ungenauigkeit der Richtungsbestimmung ist bei Methode~1 auch eine höhere Sensibilität auf Handbewegungen zu spüren, obwohl sie dieselben \textit{Deadzones} verwenden.
Diese Sensibilität ist vermutlich auf die Distanz zwischen den Markern zurückzuführen.
Die grössere Spannweite der Marker in Methode~2 ermöglicht präzisere Richtungsänderungen.

Beim Gebrauch der zweiten Methode endet der Flug bereits nach \si{90} Sekunden. 
Bei der gewählten Geschwindigkeit von \SI{0.5}{\meter\per\second} und ohne Beschleunigungsfunktionalität sind das \si{18} Sekunden mehr als die theoretisch mögliche Bestzeit.\footnotemark{}
\footnotetext{Beide Methoden verwendeten dieselben Geschwindigkeitswerte.}
Nach drei weiteren Runden pro Methode lassen sich die durchschnittlichen Rennzeiten berechnen:

\begin{itemize}
    \item \textbf{Methode~1:} \si{117} Sekunden
    \item \textbf{Methode~2:} \si{73} Sekunden
\end{itemize}

Zum Schluss wurde das Feedback der vier Testpersonen ausgewertet.
Als Anwortmöglichkeiten wurde \enquote{intuitiv}, \enquote{eher intuitiv} und \enquote{nicht intuitiv} vorgegeben.
In folgender Tabelle sind die Antworten aufgeführt:

\begin{table}[H]
    \centering
    \begin{tabular}{l|l}
        \textbf{Testende Person} & \textbf{Einschätzung nach Verwendung} \\ \hline
        Nummer 1 & eher intuitiv \\
        Nummer 2 & nicht intuitiv \\
        Nummer 3 & intuitiv \\
        Nummer 4 & eher intuitiv \\
        \hline
    \end{tabular}
    \caption{Feedback der Testpersonen}
        \label{tab:fdbk}
\end{table}

\section{Zusammenfassung der Ergebnisse}
Die durchgeführten Tests zeigen, dass das System grundsätzlich zuverlässig arbeitet und eine markerbasierte Gestensteuerung der Crazyflie 2.0 möglich ist.
Die Gestenerkennung erwies sich als stabil, solange die Marker nicht stark geneigt oder überbelichtet waren.
Die beste Erkennung wurde bei einer Entfernung von rund \SI{20}{\centi\meter} und schwacher Umgebungsbeleuchtung erreicht.
Die Drohnensteuerung reagierte präzise auf übertragene Befehle, wobei durch Funkstörungen oder unebene Untergründe gelegentlich Stabilitätsprobleme auftraten.
Eine messbare Verzögerung der Reaktionszeit lag nicht vor.
Im kombinierten Testlauf schnitt Methode~2 deutlich besser ab.
Sie erlaubte präzisere Richtungsänderungen und eine kürzere Gesamtlaufzeit von durchschnittlich \si{73} Sekunden gegenüber \si{117} Sekunden bei Methode~1.
Das Feedback der Testpersonen bestätigt diesen Eindruck:
Drei von vier Personen empfanden die Steuerung als intuitiv oder eher intuitiv.

Insgesamt zeigt sich, dass die zweite Methode sowohl technisch als auch in der subjektiven Bewertung überlegen ist und sich für eine gestenbasierte Steuerung der Crazyflie besser eignet.

% End font-size
\endgroup